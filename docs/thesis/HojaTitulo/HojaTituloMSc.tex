%\newpage
%\setcounter{page}{1}
\begin{center}
\begin{figure}
\centering%
\epsfig{file=HojaTitulo/EscudoUN,scale=1}%
\end{figure}
\thispagestyle{empty} \vspace*{2.0cm} \textbf{\huge
Separation of primary particles in a LHCb-like EM calorimeter}\\[6.0cm]
\Large\textbf{Cindy Catalina Moreno Sarria}\\[6.0cm]
\small Universidad Nacional de Colombia\\
Facultad de Ciencias, Departamento de Física\\
Bogotá, Colombia\\
A\~{n}o 2022\\
\end{center}

\newpage{\pagestyle{empty}\cleardoublepage}

\newpage
\begin{center}
\thispagestyle{empty} \vspace*{0cm} \textbf{\huge
Separation of primary particles in a LHCb-like EM calorimeter}\\[2.5cm]
\Large\textbf{Cindy Catalina Moreno Sarria}\\[2.5cm]
\small Trabajo de grado presentado como requisito parcial para optar al
t\'{\i}tulo de:\\
\textbf{Pregrado en Física}\\[2.0cm]
Directores:\\
Ph.D. Carlos Eduardo Sandoval Usme\\
Ph.D. Diego Alejandro Milanés Carreño\\[2.0cm]
L\'{\i}nea de Investigaci\'{o}n:\\
Física de Altas Energías\\
Grupo de Investigaci\'{o}n:\\
FENYX\\[2.0cm]
Universidad Nacional de Colombia\\
Facultad de Ciencias, Departamento de Física\\
Bogotá, Colombia\\
A\~{n}o 2022\\
\end{center}

\newpage{\pagestyle{empty}\cleardoublepage}

\newpage
\thispagestyle{empty} \textbf{}\normalsize


% \textbf{(Dedicatoria o un lema)}\\[4.0cm]

\begin{flushright}
\begin{minipage}{8cm}
    \noindent
        \small
        \textit{To every queer kid and girls wanting to work in STEM.}
\end{minipage}
\end{flushright}

\newpage{\pagestyle{empty}}

\newpage
\thispagestyle{empty} \textbf{}\normalsize


\textbf{\LARGE Acknowledgment}
\addcontentsline{toc}{chapter}{Acknowledgment}

I want to especially thank my professors, Diego Milanés and Carlos Sandoval,
for all the opportunities and guidance they have given me on my path to
becoming a researcher.

I also would like to give thanks to my mom, my biggest supporter, for always
believing in me and being by my side no matter what.

\newpage{\pagestyle{empty}\cleardoublepage}

\newpage
\textbf{\LARGE Resumen}

Se utilizaron cinco clasificadores multidimensionales diferentes
(MultinomialNB, BernoulliNB, Perceptron, SGDClassifier y
PassiveAggressiveClassifier) para probar la capacidad de separación de
partículas primarias en un calorímetro electromagnético similar al implementado
en el LHCb, utilizando datos de muestreo de simulaciones en Geant4. Los
clasificadores debían diferenciar entre fotones, electrones y piones neutros,
dado el número de electrones y fotones creados en las placas de centelleador y
de plomo del calorímetro. Sin embargo, no mostraron un buen rendimiento en esta
tarea, teniendo una precisión de 0,45 para MultinomialNB 0,55 para BernoulliNB,
0,33 para Perceptron, 0,33 para SGDClassifier y 0,33 para
PassiveAggressiveClassifier.

\textbf{\LARGE Abstract}

\addcontentsline{toc}{chapter}{Abstract}

Five different multidimensional classifiers (MultinomialNB, BernoulliNB,
Perceptron, SGDClassifier and PassiveAggressiveClassifier) were used to test
the capability of primary particle discrimination in an LHCb-like
electromagnetic calorimeter using sampling data from Geant4 simulation. The
classifiers were supposed to differentiate between photons, electrons and
neutral pions, given the number of electrons and photons created in the
scintillator and lead plates in the calorimeter. They however did not show a
good performance in this task, having a accuracy of 0.45 for MultinomialNB,
0.55 for BernoulliNB, 0.33 for Perceptron, 0.33 for SGDClassifier and 0.33 for
PassiveAggressiveClassifier.

\textbf{\small Keywords: Electromagnetic Calorimeter, Simulation, Geant, Particle Physics}\\

\newpage{}

\textbf{\LARGE Code Preservation}
\addcontentsline{toc}{chapter}{Code Preservation}

All code is at \url{https://github.com/afmorenosa/G4_Tesis}.
