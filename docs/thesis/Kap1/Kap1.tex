\chapter{Introduction}
In particles physics, calorimeters are the most important instrument in the
measurement of charged and neutral particles produced in the
accelerator\cite{zhu2003calorimetry}. The basis of the calorimeters is to use
the interaction of the particle with the material. This interaction produces a
shower of secondary particles, that will be absorbed by the material. The
energy absorbed due to these particles can be
measured\cite{fabjan2003calorimetry}.

Calorimeters can be divided into electromagnetic calorimeters, that detects
electrons and photos since they interact through the electromagnetic force, and
hadronic calorimeters, that use the strong and electromagnetic interactions to
detect hadrons\cite{fabjan2003calorimetry}. For this thesis, we will focus on
the electromagnetic calorimeters. Another classification of the calorimeters is
due to their construction. Then it can be a sampling calorimeter if it is built
with alternating layers of absorbent material and an active medium, i.e., a
scintillator. A calorimeter also can be homogeneous, in that case it is built
with just one material which produce the shower and detects the absorbed
energy\cite{fabjan2003calorimetry}.

The LHCb experiment use the high rate of B particles production to study the
CP-symmetry violation\cite{omelaenko2000lhcb, Collaboration_2008}. The
electromagnetic calorimeter provides transverse energy for electrons and
photons candidates that is used in the first level
trigger\cite{Antunes-Nobrega:630828}, that helps to reconstruct neutral pions
(\(\pi^0\)) and prompt photons.

The electromagnetic calorimeter is divided in three sections (inner, middle and
outer section)\cite{omelaenko2000lhcb}, each one composed by a mesh of
individual modules with similar characteristics. Since this electromagnetic
calorimeter is a sampling calorimeter, the modules are a combination of
alternating square lead, that works as an absorbent
material\cite{omelaenko2000lhcb}, and scintillator tiles, which produce the
shower of secondary particles. The size of each plate is
\(\SI{121.2}{\milli\metre}\), covered by steel sheets. All the tiles have a
hole patter, so WLS fibers, corresponding to the readout of the module, pass
through it\cite{omelaenko2000lhcb, Machikhiliyan_2009, Collaboration_2008}.

The differences between the modules of each section is the number of readout
cells. The modules of outer section have only one readout cell. In the middle
section, the modules hold 4 readout cells, corresponding to a square of size
\(\SI{60.6}{\milli\metre}\). Finally, the modules in the inner section have
nine readout cells, having the greater granularity in the
calorimeter\cite{omelaenko2000lhcb}.

As part of the main purposes of the electromagnetic, it should be able to
detect photons that allow to reconstruct B-decays, with prompt photons or
\(\pi^0\) in the final state\cite{omelaenko2000lhcb, Boldyrev_2020}. Although
the granularity of the calorimeter have work well for a luminosity up to
\(\SI{2e33}{\per\centi\metre\per\second}\)\cite{guz2013lhcb}, the
identification of high energetic \(\pi^0\) events is not optimal. The neutral
pion decays into two quite parallel photons, that can be misinterpreted as a
single photon with high energy. An alternative to the primary particle
identification is the implementation of machine learning algorithms that use
the information of the electromagnetic shower produced inside the materials to
discriminate each particle\cite{Boldyrev_2020}. This work implements various
multidimensional classifiers to test the incident particle differentiation,
using sampling data from Geant4 simulations.
